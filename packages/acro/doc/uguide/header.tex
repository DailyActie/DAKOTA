\documentclass[11pt]{article}

%\documentclass[10pt]{book}
\newcommand{\chaptermark}{1}{}

\usepackage{makeidx}
\usepackage{fancyhdr}
\usepackage{graphicx}
\usepackage{multicol}
\usepackage{float}
\usepackage{alltt}
\usepackage{doxygen}
\setcounter{tocdepth}{1}
%\makeindex
%\setlength{\footrulewidth}{0.4pt}



\setlength{\topmargin}{-.25in}
\setlength{\textheight}{9in}
\setlength{\oddsidemargin}{0.0in}
\setlength{\evensidemargin}{0.0in}
\setlength{\textwidth}{6.5in}

\setlength{\headwidth}{6.5in}
%\setlength{\footrulewidth}{0pt}
\setlength{\parindent}{3ex}
\setlength{\parsep}{0pt}

\setcounter{tocdepth}{1}

\pagestyle{plain}

\begin{document}

\lhead{}
\rhead{}
\rfoot{}
\lfoot{}
\cfoot{}
\chead{}

%%
%% Title Page
%%

\begin{center}
SANDXXXX--XXXX

Unlimited Release

Printed December XXXX
\end{center}

\vspace{0.8in}

\begin{center}
{\bf \LARGE Acro User Manual\\
[1ex]\Large Version 1.0}
\vspace*{0.4in}

William E. Hart\\
Discrete Algorithms and Mathematics Department\\
Sandia National Laboratories\\
P. O. Box 5800\\
Albuquerque, NM\\
{\tt http://www.cs.sandia.gov/$\sim$wehart}\\
{\tt wehart@sandia.gov}\\
\end{center}

\vspace*{1.0in}

\begin{quote}
{\bf\large Abstract}
PEBBL is a C++ framework for implementing general parallel
branch-and-bound optimization algorithms, providing a mechanism for
the efficient implementation of a wide range of branch-and-bound
methods on an large variety of parallel computing platforms.
This document describes:
\begin{itemize}
\item The history, goals, and general properties of PEBBL
\item How to download and compile PEBBL
\item PEBBL's special search capabilities, including enumeration,
  early output, and checkpointing
\item PEBBL's basic architecture, including its serial and parallel layers
\item The design of the serial layer and the notion of manipulating
  subproblem states
\item The design and capabilities of the parallel layer
\item How to build a simple serial branch-and-bound algorithm using
  PEBBL
\item How to extend a serial implementation into a parallel one
\item Many of the numerous parameters that can be used to 
tune PEBBL's behavior.
\end{itemize}

\end{quote}

\vfill
\newpage

\pagestyle{fancy}
\begin{quote}
{\em acro} - From Greek akros, at the point, end, or top.\\

{\em acro} - A generic term for warblers of the genus {\em Acrocephalus}, usually refering to the sedge and/or reed warblers. The sedge warbler is a small, quite plump, warbler with a striking broad creamy stripe above its eye, and greyish brown legs. It is brown above with blackish streaks and creamy white underneath. It spends summers in the UK and winters in Africa, south of the Sahara Desert. Its song is a noisy, rambling warble compared to the more rhythmic song of the reed warbler.\\

{\em acro} - A Common Repository for Optimizers
\end{quote}

\vfill
\newpage


\pagenumbering{roman}
%\tableofcontents

\clearemptydoublepage
\pagenumbering{arabic}
\setcounter{page}{3}
\cfoot{\thepage}


\section{Introduction}

The Acro Project is an effort to facilitate the design, development, integration and support of optimization software libraries. The goal of the Acro project is to develop optimization solvers and libraries using object-oriented software frameworks that facilitate the application of these solvers to large-scale engineering and scientific applications. Thus Acro includes both individual optimization solvers as well as optimization frameworks that provide abstract interfaces for flexible interoperability of solver components.

This document describes the Acro command line interface (ACLI), which provides a mechanism for defining and solving optimization problems with a general XML syntax.  The goal of the ACLI is to provide a simple, flexible interface for the optimization solvers in Acro. Acro integrates a variety of optimization software packages, including both libraries developed at Sandia National Laboratories as well as publicly available third-party libraries. Thus, the ACLI provides a single framework that supports a wide range of optimization methods. This interface is designed to be easy for users.  Further, the XML driver syntax provides generic mechanisms that should simplify the integration of the ACLI into third-party applications.

The ACLI is designed to be used in two different ways: (1) as an AMPL solver interface, and (2) as an interface for external applications. The next sections describe these usage models and provide examples for the use of ACLI. The remainder of this document describes the XML syntax used to drive ACLI (outside of AMPL).  This XML syntax allows the user to define, reformulate and solve optimization problems.


\newpage

\input{ea-doc}
\newpage

\input{patternsearch-doc}
\newpage

\input{soliswets-doc}
\newpage

\input{smc-doc}
\newpage

\input{install}
\newpage

%\input{refs}

%\begin{center}
{\Large \bf Acknowledgements}
\end{center}

A number of people have contributed to the JEGA project in various
ways.  Special thanks go out to Dr. Laura Swiler at Sandia National
Laboratories.  She is ultimately responsible for the existence of
JEGA and has provided and continues to provide valuable technical
guidance.

Thanks go out to the rest of the DAKOTA development team
who have all provided technical guidance during the development of
JEGA.



%\newpage
%\bibliographystyle{abbrv}
%\bibliography{hart}

\end{document}

END-OF-DOC



