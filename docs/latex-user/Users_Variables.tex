\chapter{Variables}\label{variables}

\section{Overview}\label{variables:overview}

The \texttt{variables} specification in a DAKOTA input file specifies
the parameter set to be iterated by a particular method. In the case
of an optimization study, these variables are adjusted in order to
locate an optimal design; in the case of parameter studies/sensitivity
analysis/design of experiments, these parameters are perturbed to
explore the parameter space; and in the case of uncertainty analysis,
the variables are associated with distribution/interval
characterizations which are used to compute corresponding
distribution/interval characterizations for response functions. To
accommodate these and other types of studies, DAKOTA supports design,
uncertain, and state variable types for continuous and discrete
variable domains, where uncertain types can be further categorized as
either aleatory or epistemic and discrete domains can be further
categorized as discrete range, discrete integer set, or discrete real
set.

This chapter will present a brief overview of the types of variables
and their uses, as well as cover some user issues relating to file
formats and the active set vector.  For a detailed description of
variables section syntax and example specifications, refer to the
Variables Commands chapter in the DAKOTA Reference Manual~\cite{RefMan}.


\section{Design Variables}\label{variables:design}


Design variables are those variables which are modified for the
purposes of computing an optimal design. These variables may be
continuous (real-valued between bounds), discrete range
(integer-valued between bounds), discrete set of integers
(integer-valued from finite set), and discrete set of reals
(real-valued from finite set).

\subsection{Continuous Design Variables}\label{variables:design:cdv}

The most common type of design variables encountered in engineering
applications are of the continuous type. These variables may assume
any real value (e.g., \texttt{12.34}, \texttt{-1.735e+07}) within
their bounds. All but a handful of the optimization algorithms in
DAKOTA support continuous design variables exclusively.

\subsection{Discrete Design Variables}\label{variables:design:ddv}

Engineering design problems may contain discrete variables such as
material types, feature counts, stock gauge selections, etc. These
variables may assume only a fixed number of values, as compared to
a continuous variable which has an uncountable number of possible 
values within its range.  Discrete variables may involve a range 
of consecutive integers ($x$ can be any integer between 
\texttt{1} and \texttt{10}), a set of integer values ($x$ can 
be \texttt{101}, \texttt{212}, or \texttt{355}), or a set of real 
values (e.g., $x$ can be \texttt{4.2}, \texttt{6.4}, or \texttt{8.5}).

Discrete variables may be classified as either ``categorical'' or
``noncategorical.''  In the latter noncategorical case, the discrete
requirement can be relaxed during the solution process since the model
can still compute meaningful response functions for values outside the
allowed discrete range or set. For example, a discrete variable
representing the thickness of a structure is generally a
noncategorical variable since it can assume a continuous range of
values during the algorithm iterations, even if it is desired to have
a stock gauge thickness in the end. In the former categorical case,
the discrete requirement cannot be relaxed since the model cannot
obtain a solution for values outside the range or set. For example,
feature counts are generally categorical discrete variables, since
most computational models will not support a non-integer value for the
number of instances of some feature (e.g., number of support brackets).

Gradient-based optimization methods cannot be directly applied to
problems with discrete variables since derivatives only exist for a
variable continuum. For problems with noncategorical variables, branch
and bound techniques can be used to relax the discrete requirements
and apply gradient-based methods to a series of generated
subproblems. For problems with categorical variables,
nongradient-based methods (e.g., \texttt{coliny\_ea}) are commonly
used. Branch and bound techniques are discussed in
Section~\ref{strat:minlp} and nongradient-based methods are further
described in Chapter~\ref{opt}.

In addition to engineering applications, many non-engineering
applications in the fields of scheduling, logistics, and resource
allocation contain discrete design parameters. Within the Department
of Energy, solution techniques for these problems impact programs in
stockpile evaluation and management, production planning,
nonproliferation, transportation (routing, packing, logistics),
infrastructure analysis and design, energy production, environmental
remediation, and tools for massively parallel computing such as domain
decomposition and meshing.

\subsubsection{Discrete Design Integer Variables}\label{variables:design:ddiv}

There are two types of discrete design integer variables supported by
DAKOTA.
\begin{itemize}

\item  The \texttt{discrete\_design\_range} specification supports a
range of consecutive integers between specified \texttt{lower\_bounds}
and \texttt{upper\_bounds}.

\item The \texttt{discrete\_design\_set\_integer} specification supports 
a set of enumerated integer values through the \texttt{set\_values} 
specification.  The set of values specified is stored internally as an 
STL set container, which enforces an ordered, unique representation of 
the integer data.  Underlying this set of ordered, unique integers is a 
set of indices that run from 0 to one less than the number of set values.
These indices are used by some iterative algorithms (e.g., parameter 
studies, COLINY iterators) for simplicity in discrete value enumeration 
when the actual corresponding set values are immaterial.  In the case of 
parameter studies, this index representation is exposed through certain 
step and partition control specifications (see Chapter~\ref{ps}).

\end{itemize}

\subsubsection{Discrete Design Real Variables}\label{variables:design:ddrv}

There is one type of discrete design real variable supported by
DAKOTA.
\begin{itemize}

\item The \texttt{discrete\_design\_set\_real} specification
specification supports a set of enumerated real values through the
\texttt{set\_values} specification.  As for the discrete integer
set variables described in Section~\ref{variables:design:ddiv},
internal storage of the set values is ordered and unique and an
underlying index representation is exposed for the specification of
some iterative algorithms.

\end{itemize}


\section{Uncertain Variables}\label{variables:uncertain}


Deterministic variables (i.e., those with a single known value) do not
capture the behavior of the input variables in all situations. In many
cases, the exact value of a model parameter is not precisely known. An
example of such an input variable is the thickness of a heat treatment
coating on a structural steel I-beam used in building construction.
Due to variabilities and tolerances in the coating process, the
thickness of the layer is known to follow a normal distribution with a
certain mean and standard deviation as determined from experimental
data. The inclusion of the uncertainty in the coating thickness is
essential to accurately represent the resulting uncertainty in the
response of the building.

\subsection{Aleatory Uncertain Variables}\label{variables:uncertain:auv}

Aleatory uncertainties are irreducible variabilities inherent in
nature.  They are characterized by having a sufficiently rich set of
data as to allow modeling using probability distributions, and
probabilistic methods are commonly used for propagating nput aleatory
uncertainties described by probability distribution specifications.
The two following sections describe the continuous and discrete
aleatory uncertain variables supported by DAKOTA.

For aleatory random variables, DAKOTA supports a user-supplied
correlation matrix to provide correlations among the input
variables. By default, the correlation matrix is set to the identity
matrix, i.e., no correlation among the uncertain variables.

For additional information on random variable probability
distributions, refer to~\cite{Hal00} and~\cite{Swi04}. Refer to the DAKOTA
Reference Manual~\cite{RefMan} for more detail on the uncertain variable
specifications and to Chapter~\ref{uq} for a description of methods
available to quantify the uncertainty in the response.

\subsubsection{Continuous Aleatory Uncertain Variables}\label{variables:uncertain:cauv}

\begin{itemize}

\item Normal: a probability distribution characterized by a mean and 
  standard deviation. Also referred to as Gaussian. Bounded normal is
  also supported by some methods with an additional specification of
  lower and upper bounds.

\item Lognormal: a probability distribution characterized by a mean 
  and either a standard deviation or an error factor. The natural
  logarithm of a lognormal variable has a normal distribution. Bounded
  lognormal is also supported by some methods with an additional
  specification of lower and upper bounds.

\item Uniform: a probability distribution characterized by a lower bound 
  and an upper bound.  Probability is constant between the bounds.

\item Loguniform: a probability distribution characterized by a lower 
  bound and an upper bound.  The natural logarithm of a loguniform
  variable has a uniform distribution.

\item Triangular: a probability distribution characterized by a mode, a 
  lower bound, and an upper bound.

\item Exponential: a probability distribution characterized by a beta parameter.

\item Beta: a flexible probability distribution characterized by a lower 
  bound and an upper bound and alpha and beta parameters.  The uniform 
  distribution is a special case.

\item Gamma: a flexible probability distribution characterized by alpha 
  and beta parameters.  The exponential distribution is a special case.

\item Gumbel: the Type I Largest Extreme Value probability distribution.  
  Characterized by alpha and beta parameters.

\item Frechet: the Type II Largest Extreme Value probability distribution.  
  Characterized by alpha and beta parameters.

\item Weibull: the Type III Smallest Extreme Value probability distribution.  
  Characterized by alpha and beta parameters.

\item Histogram Bin: an empirically-based probability distribution 
  characterized by a set of $(x,y)$ pairs that map out
  histogram bins (a continuous interval with associated bin count).

\end{itemize}

\subsubsection{Discrete Aleatory Uncertain Variables}\label{variables:uncertain:dauv}

The following types of discrete aleatory uncertain variables are available:

\begin{itemize}

\item Poisson: integer-valued distribution used to predict the number of 
  discrete events that happen in a given time interval.

\item Binomial: integer-valued distribution used to predict 
  the number of failures in a number of independent tests or trials.

\item Negative Binomial: integer-valued distribution used to predict the
  number of times to perform a test to have a target number of successes.

\item Geometric: integer-valued distribution used to model the number of 
  successful trials that might occur before a failure is observed.

\item Hypergeometric: integer-valued distribution used to model the number 
  of failures observed in a set of tests that has a known proportion of 
  failures.

\item Histogram Point: an empirically-based probability distribution 
  characterized by a set of real-valued $(x,y)$ pairs that map out
  histogram points (a discrete point value with associated count).

\end{itemize}


\subsection{Epistemic Uncertain Variables}\label{variables:uncertain:euv}

Epistemic uncertainties are reducible uncertainties resulting from a
lack of knowledge.  For epistemic uncertainties, data is generally
sparse, making the use of probability theory questionable and leading
to nonprobabilistic methods based on interval specifications  DAKOTA
currently supports one epistemic uncertain variable type.

\subsubsection{Continuous Epistemic Uncertain Variables}\label{variables:uncertain:ceuv}

\begin{itemize}

\item Interval: an interval-based specification characterized by sets of
  lower and upper bounds and Basic Probability Assignments (BPAs)
  associated with each interval.  The intervals may be overlapping,
  contiguous, or disjoint, and a single interval (with probability =
  1) per variable is an important special case.  The interval
  distribution is not a probability distribution, as the exact
  structure of the probabilities within each interval is not known.
  It is commonly used with epistemic uncertainty methods.

\end{itemize}

\section{State Variables}\label{variables:state}

State variables consist of ``other'' variables which are to be mapped
through the simulation interface, in that they are not to be used for
design and they are not modeled as being uncertain. State variables
provide a convenient mechanism for parameterizing additional model
inputs which, in the case of a numerical simulator, might include
solver convergence tolerances, time step controls, or mesh fidelity
parameters. For additional model parameterizations involving strings
(e.g., ``mesh1.exo''), refer to the analysis components specification
described in Section~\ref{variables:parameters:standard} and in the
Interface Commands chapter of the DAKOTA Reference
Manual~\cite{RefMan}.  Similar to the design variables discussed in
Section~\ref{variables:design}, state variables can be a continuous
range (real-valued between bounds), a discrete range (integer-valued
between bounds), a discrete integer-valued set, or a discrete
real-valued set.

State variables, as with other types of variables, are viewed
differently depending on the method in use. Since these variables are
neither design nor uncertain variables, algorithms for optimization,
least squares, and uncertainty quantification do not iterate on these
variables; i.e., they are not active and are hidden from the
algorithm. However, DAKOTA still maps these variables through the
user's interface where they affect the computational model in use.
This allows optimization, least squares, and uncertainty
quantification studies to be executed under different simulation
conditions (which will result, in general, in different results).
Parameter studies and design of experiments methods, on the other
hand, are general-purpose iterative techniques which do not draw a
distinction between variable types. They include state variables in
the set of variables to be iterated, which allows these studies to
explore the effect of state variable values on the response data of
interest.

In the future, state variables might be used in direct coordination
with an optimization, least squares, or uncertainty quantification
algorithm. For example, state variables could be used to enact model
adaptivity through the use of a coarse mesh or loose solver tolerances
in the initial stages of an optimization with continuous model
refinement as the algorithm nears the optimal solution.

\section{Mixed Variables}\label{variables:mixed}

As alluded to in the previous section, the iterative method selected
for use in DAKOTA determines what subset, or view, of the variables
data is active in the iteration. The general case of having a mixture
of various different types of variables is supported within all of the
DAKOTA methods even though certain methods will only modify certain
types of variables (e.g., optimizers and least squares methods only
modify design variables, and uncertainty quantification methods, with
the exception of \texttt{all\_variables} mode, only utilize uncertain
variables).  This implies that variables which are not under the
direct control of a particular iterator will be mapped through the
interface in an unmodified state. This allows for a variety of
parameterizations within the model in addition to those which are
being used by a particular iterator, which can provide the convenience
of consolidating the control over various modeling parameters in a
single file (the DAKOTA input file). An important related point is
that the variable set that is active with a particular iterator is the
same variable set for which derivatives are typically computed (see
Section~\ref{responses:active}).

\section{DAKOTA Parameters File Data Format}\label{variables:parameters}

Simulation interfaces which employ system calls and forks to create
separate simulation processes must communicate with the simulation
code through the file system. This is accomplished through the reading
and writing of parameters and results files. DAKOTA uses a particular
format for this data input/output. Depending on the user's interface
specification, DAKOTA will write the parameters file in either
standard or APREPRO format (future XML formats are planned). The
former option uses a simple ``\texttt{value tag}'' format, whereas the
latter option uses a ``\texttt{\{ tag = value \}}'' format for
compatibility with the APREPRO utility~\cite{Sja92} (as well as
DPrePro, BPREPRO, and JPrePost variants).

\subsection{Parameters file format (standard)}\label{variables:parameters:standard}

Prior to invoking a simulation, DAKOTA creates a parameters file which
contains the current parameter values and a set of function requests.
The standard format for this parameters file is shown in
Figure~\ref{variables:figure01}.

\begin{figure}
  \centering
  \begin{bigbox}
  \begin{alltt}
    <int>    variables
    <double> <label_cdv\(\sb{i}\)>         (i = 1 to n\(\sb{cdv}\))
    <int>    <label_ddiv\(\sb{i}\)>        (i = 1 to n\(\sb{ddiv}\))
    <double> <label_ddrv\(\sb{i}\)>        (i = 1 to n\(\sb{ddrv}\))
    <double> <label_cauv\(\sb{i}\)>        (i = 1 to n\(\sb{cauv}\))
    <int>    <label_dauiv\(\sb{i}\)>       (i = 1 to n\(\sb{dauiv}\))
    <double> <label_daurv\(\sb{i}\)>       (i = 1 to n\(\sb{daurv}\))
    <double> <label_ceuv\(\sb{i}\)>        (i = 1 to n\(\sb{ceuv}\))
    <int>    <label_deuiv\(\sb{i}\)>       (i = 1 to n\(\sb{deuiv}\))
    <double> <label_deurv\(\sb{i}\)>       (i = 1 to n\(\sb{deurv}\))
    <double> <label_csv\(\sb{i}\)>         (i = 1 to n\(\sb{csv}\))
    <int>    <label_dsiv\(\sb{i}\)>        (i = 1 to n\(\sb{dsiv}\))
    <double> <label_dsrv\(\sb{i}\)>        (i = 1 to n\(\sb{dsrv}\)) \color{blue}
    <int>    functions
    <int>    ASV_i:label_response\(\sb{i}\)       (i = 1 to m) \color{red}
    <int>    derivative_variables
    <int>    DVV_i:label_cdv\(\sb{i}\)            (i = 1 to p) \color{green}
    <int>    analysis_components
    <string> AC_i:analysis_driver_name\(\sb{i}\)  (i = 1 to q)
  \end{alltt}
  \end{bigbox}
  \caption{Parameters file data format - standard option.}
  \label{variables:figure01}
\end{figure}

where ``\texttt{<int>}'' denotes an integer value,
``\texttt{<double>}'' denotes a double precision value, and
``\texttt{<string>}'' denotes a string value. Each of the colored
blocks (black for variables, blue for active set vector, red for
derivative variables vector, and green for analysis components)
denotes an array which begins with an array length and a descriptive
tag.  These array lengths are useful for dynamic memory allocation
within a simulator or filter program.

The first array for variables begins with the total number of
variables (\texttt{n}) with its identifier string
``\texttt{variables}.''  The next \texttt{n} lines specify the current
values and descriptors of all of the variables within the parameter
set \emph{in the following order}: continuous design, discrete integer
design (integer range, integer set), discrete real design (real set),
continuous aleatory uncertain (normal, lognormal, uniform, loguniform,
triangular, exponential, beta, gamma, gumbel, frechet, weibull,
histogram bin), discrete integer aleatory uncertain (poisson,
binomial, negative binomial, geometric, hypergeometric), discrete real
aleatory uncertain (histogram point), continuous epistemic uncertain
(interval), discrete integer epistemic uncertain (none at this time),
discrete real epistemic uncertain (none at this time), continuous
state, discrete integer state (integer range, integer set), and
discrete real state (real set) variables. This ordering is consistent
with the lists in
Sections~\ref{variables:design:ddiv}, \ref{variables:uncertain:cauv}
and~\ref{variables:uncertain:dauv}
and the specification order in dakota.input.txt.  The lengths of these
vectors add to a total of $n$ (that is, $n = n_{cdv} + n_{ddiv} +
n_{ddrv} + n_{cauv} + n_{dauiv} + n_{daurv} + n_{ceuv} + n_{deuiv} +
n_{deurv} + n_{csv} + n_{dsiv} + n_{dsrv}$).  If any of the variable
types are not present in the problem, then its block is omitted
entirely from the parameters file.  The tags are the variable
descriptors specified in the user's DAKOTA input file, or if no
descriptors have been specified, default descriptors are used.

The second array for the active set vector (ASV) begins with the total
number of functions (\texttt{m}) and its identifier string
``\texttt{functions}.'' The next \texttt{m} lines specify the request
vector for each of the \texttt{m} functions in the response data set
followed by the tags ``\texttt{ASV\_i:label\_response}'', where the
label is either a user-provided response descriptor or a
default-generated one. These integer codes indicate what data is
required on the current function evaluation and are described further
in Section~\ref{variables:asv}.

The third array for the derivative variables vector (DVV) begins with
the number of derivative variables (\texttt{p}) and its identifier
string ``\texttt{derivative\_variables}.'' The next \texttt{p} lines
specify integer variable identifiers followed by the tags
``\texttt{DVV\_i:label\_cdv}''.  These integer identifiers are used to
identify the subset of variables that are active for the calculation
of derivatives (gradient vectors and Hessian matrices), and correspond
to the list of variables in the first array (e.g., an identifier of 2
indicates that the second variable in the list is active for
derivatives).  The labels are again taken from user-provided or
default variable descriptors.

The final array for the analysis components (AC) begins with the
number of analysis components (\texttt{q}) and its identifier string
``\texttt{analysis\_components}.'' The next \texttt{q} lines provide
additional strings for use in specializing a simulation interface
followed by the tags ``\texttt{AC\_i:analysis\_driver\_name}'', where
\texttt{analysis\_driver\_name} indicates the driver associated with
this component.  These strings are specified in a user's input file
for a set of \texttt{analysis\_drivers} using the
\texttt{analysis\_components} specification.  The subset of the
analysis components used for a particular analysis driver is the set
passed in a particular parameters file.

Several standard-format parameters file examples are shown in
Section~\ref{interfaces:mappings}.

\subsection{Parameters file format (APREPRO)}\label{variables:parameters:aprepro}

For the APREPRO format option, the same data is present and the same
ordering is used as in the standard format. The only difference is
that values are associated with their tags within ``\texttt{\{ tag =
value \}}'' constructs as shown in Figure~\ref{variables:figure02}.
An APREPRO-format parameters file example is shown in
Section~\ref{interfaces:mappings}.

The use of the APREPRO format option allows direct usage of these
parameters files by the APREPRO utility, which is a file pre-processor
that can significantly simplify model parameterization.  Similar
pre-processors include DPrePro, BPREPRO, and JPrePost.  \emph{[Note:
APREPRO is a Sandia-developed pre-processor that is not currently
distributed with DAKOTA.  DPrePro is a Perl script distributed with
DAKOTA that performs many of the same functions as APREPRO, and is
optimized for use with DAKOTA parameters files in either format.
BPREPRO and JPrePost are additional Perl and JAVA tools, respectively,
in use at other sites.]}  When a parameters file in APREPRO format is
included within a template file (using an include directive), the
APREPRO utility recognizes these constructs as variable definitions
which can then be used to populate targets throughout the template
file~\cite{Sja92}.  DPrePro, conversely, does not require the use of
includes since it processes the DAKOTA parameters file and template
simulation file separately to create a simulation input file populated
with the variables data.

\begin{figure}
  \begin{bigbox}
  \centering
  \begin{alltt}
    \{ DAKOTA_VARS = <int> \}
    \{ <label_cdv\(\sb{i}\)> = <double> \}         (i = 1 to n\(\sb{cdv}\))
    \{ <label_ddiv\(\sb{i}\)> = <int> \}           (i = 1 to n\(\sb{ddiv}\))
    \{ <label_ddrv\(\sb{i}\)> = <double> \}        (i = 1 to n\(\sb{ddrv}\))
    \{ <label_cauv\(\sb{i}\)> = <double> \}        (i = 1 to n\(\sb{cauv}\))
    \{ <label_dauiv\(\sb{i}\)> = <int> \}          (i = 1 to n\(\sb{dauiv}\))
    \{ <label_daurv\(\sb{i}\)> = <double> \}       (i = 1 to n\(\sb{daurv}\))
    \{ <label_ceuv\(\sb{i}\)> = <double> \}        (i = 1 to n\(\sb{ceuv}\))
    \{ <label_deuiv\(\sb{i}\)> = <int> \}          (i = 1 to n\(\sb{deuiv}\))
    \{ <label_deurv\(\sb{i}\)> = <double> \}       (i = 1 to n\(\sb{deurv}\))
    \{ <label_csv\(\sb{i}\)> = <double> \}         (i = 1 to n\(\sb{csv}\))
    \{ <label_dsiv\(\sb{i}\)> = <int> \}           (i = 1 to n\(\sb{dsiv}\))
    \{ <label_dsrv\(\sb{i}\)> = <double> \}        (i = 1 to n\(\sb{dsrv}\)) \color{blue}
    \{ DAKOTA_FNS = <int> \}
    \{ ASV_i:label_response\(\sb{i}\) = <int> \}              (i = 1 to m) \color{red}
    \{ DAKOTA_DER_VARS = <int> \}
    \{ DVV_i:label_cdv\(\sb{i}\) = <int> \}                   (i = 1 to p) \color{green}
    \{ DAKOTA_AN_COMPS = <int> \}
    \{ AC_i:analysis_driver_name\(\sb{i}\) = <string> \}      (i = 1 to q)
  \end{alltt}
  \end{bigbox}
  \caption{Parameters file data format - APREPRO option.}
  \label{variables:figure02}
\end{figure}

\section{The Active Set Vector}\label{variables:asv}

The active set vector contains a set of integer codes, one per
response function, which describe the data needed on a particular
execution of an interface. Integer values of 0 through 7 denote a
3-bit binary representation of all possible combinations of value,
gradient, and Hessian requests for a particular function, with the
most significant bit denoting the Hessian, the middle bit denoting the
gradient, and the least significant bit denoting the value. The
specific translations are shown in Table~\ref{variables:table01}.

\begin{table}
  \centering
  \caption{Active set vector integer codes.}
  \label{variables:table01}\vspace{2mm}
  \begin{tabular}{|c|c|l|}
    \hline
    Integer Code & Binary representation & Meaning \\
    \hline
    7 & 111 & Get Hessian, gradient, and value \\
    6 & 110 & Get Hessian and gradient \\
    5 & 101 & Get Hessian and value \\
    4 & 100 & Get Hessian \\
    3 & 011 & Get gradient and value \\
    2 & 010 & Get gradient \\
    1 & 001 & Get value \\
    0 & 000 & No data required, function is inactive \\
    \hline
  \end{tabular}
\end{table}

The active set vector in DAKOTA gets its name from managing the active
set, i.e., the set of functions that are active on a particular
function evaluation. However, it also manages the type of data that is
needed for functions that are active, and in that sense, has an
extended meaning beyond that typically used in the optimization
literature.

\subsection{Active set vector control}\label{variables:asv:control}

Active set vector control may be turned off to allow the user to
simplify the supplied interface by removing the need to check the
content of the active set vector on each evaluation. The Interface
Commands chapter in the DAKOTA Reference Manual~\cite{RefMan} provides
additional information on this option (\texttt{deactivate
active\_set\_vector}).  Of course, this option trades some efficiency
for simplicity and is most appropriate for those cases in which only a
relatively small penalty occurs when computing and returning more data
than may be needed on a particular function evaluation.
