\chapter{Responses}\label{responses}

\section{Overview}\label{responses:overview}

The \texttt{responses} specification in a Dakota input file controls
the types of data that can be returned from an interface during
Dakota's execution. The specification includes the number and type of
response functions (objective functions, nonlinear constraints, calibration 
terms, etc.) as well as availability of first and second
derivatives (gradient vectors and Hessian matrices) for these response
functions.

This chapter will present a brief overview of the response data sets
and their uses, as well as cover some user issues relating to file
formats and derivative vector and matrix sizing. For a detailed
description of responses section syntax and example specifications,
refer to the Responses Commands chapter in the Dakota Reference
Manual~\cite{RefMan}.

\subsection{Response function types}\label{responses:overview:types}

The types of response functions listed in the responses
specification should be consistent with the
iterative technique called for in the method specification:

\begin{itemize}

\item an optimization data set comprised of
  \texttt{num\_objective\_functions},\\
  \texttt{num\_nonlinear\_inequality\_constraints}, and
  \texttt{num\_nonlinear\_equality\_constraints}.  This data set is
  appropriate for use with optimization methods (e.g., the methods in
  Chapter~\ref{opt}).
  
\item a calibration data set comprised of 
  \texttt{calibration\_terms},\\
  \texttt{num\_nonlinear\_inequality\_constraints}, and
  \texttt{num\_nonlinear\_equality\_constraints}.  This data set is
  appropriate for use with nonlinear least squares algorithms
  (e.g., the methods in Chapter~\ref{nls}).
  
\item a generic data set comprised of \texttt{num\_response\_functions}.  
  This data set is appropriate for use with uncertainty quantification
  methods (e.g., the methods in Section~\ref{uq}).
  
\end{itemize}

Certain general-purpose iterative techniques, such as parameter
studies and design of experiments methods, can be used with any of
these data sets.

\subsection{Gradient availability}\label{responses:overview:gradient}

Gradient availability for these response functions may be described by:

\begin{itemize}

\item \texttt{no\_gradients}: gradients will not be used.

\item \texttt{numerical\_gradients}: gradients are needed and will
  be approximated by finite differences.

\item \texttt{analytic\_gradients}: gradients are needed and will be supplied
  by the simulation code (without any finite differencing by Dakota).

\item \texttt{mixed\_gradients}: the simulation will supply some gradient
  components and Dakota will approximate the others by finite
  differences.

\end{itemize}

The gradient specification also links back to the iterative method
used. Gradients commonly are needed when the iterative
study involves gradient-based optimization, reliability analysis for
uncertainty quantification, or local sensitivity analysis.

\subsection{Hessian availability}\label{responses:overview:hessian}

Hessian availability for the response functions is similar to the
gradient availability specifications, with the addition of support
for ``quasi-Hessians":

\begin{itemize}

\item \texttt{no\_hessians}: Hessians will not be used.

\item \texttt{numerical\_gradients}: Hessians are needed and will be
  approximated by finite differences.  These finite differences may be
  involve first-order differences of gradients (if analytic gradients
  are available for the response function of interest) or second-order 
  differences of function values (in all other cases).

\item \texttt{quasi\_hessians}: Hessians are needed and will be 
  approximated by secant updates (BFGS or SR1) from a series of 
  gradient evaluations.

\item \texttt{analytic\_hessians}: Hessians are needed and are
  available directly from the simulation code.

\item \texttt{mixed\_hessians}: Hessians are needed and will be 
  obtained from a mix of numerical, analytic, and ``quasi" sources.

\end{itemize}

The Hessian specification also links back to the iterative method in
use; Hessians commonly would be used in gradient-based
optimization by full Newton methods or in reliability analysis
with second-order limit state approximations or second-order
probability integrations.

\section{Dakota Results File Data Format}\label{responses:results}

Simulation interfaces using system calls and forks to create
separate simulation processes must communicate with the simulation
through the file system. This is done by reading and
writing files of parameters and results. Dakota uses its own format
for this data input/output. For the results file, only one format is
supported (versus the two parameter-file formats described in
Section~\ref{variables:parameters}). Ordering of response functions is
as listed in Section~\ref{responses:overview:types} (e.g., objective
functions or calibration terms are first, followed by nonlinear
inequality constraints, followed by nonlinear equality constraints).

After a simulation, Dakota expects to read a file
containing responses reflecting the current parameters and
corresponding to the function requests in the active
set vector. The response data must be in the format
shown in Figure \ref{responses:figure01}.

\begin{figure}
  \centering
  \begin{bigbox}
  \begin{alltt}
    <double> <fn_tag\(\sb{1}\)>
    <double> <fn_tag\(\sb{2}\)>
    ...
    <double> <fn_tag\(\sb{m}\)> \color{blue}
    [ <double> <double> .. <double> ]
    [ <double> <double> .. <double> ]
    ...
    [ <double> <double> .. <double> ] \color{red}
    [[ <double> <double> .. <double> ]]
    [[ <double> <double> .. <double> ]]
    ...
    [[ <double> <double> .. <double> ]]
  \end{alltt}
  \end{bigbox}
  \caption{Results file data format.}
  \label{responses:figure01}
\end{figure}

The first block of data (shown in black) conveys the requested function values
and is followed by a block of requested gradients
(shown in blue), followed by a block of requested Hessians (shown
in red). If the amount of data in the file does not match the function
request vector, Dakota will abort execution with an error message.

Function values have no bracket delimiters and \emph{optionally} one
character-string tag per function can be supplied.  These tags are not
used by
Dakota and are only included as an optional field for consistency with
the parameters file format and for backwards compatibility.  If
tags are used, they must be separated from
numeric function values by white space (one or more blanks, tabs, or newline
characters) and there must not
be any white space embedded within a character-string tag (e.g., use
``\texttt{variable1}'' or ``\texttt{variable\_1},'' but not
``\texttt{variable 1}'').

Gradient vectors are surrounded by single brackets
[\ldots$n_{dvv}$-vector of doubles\ldots]. Tags are not used and must
not be present. White space separating the brackets from the data is
optional.

Hessian matrices are surrounded by double brackets
[[\ldots$n_{dvv} \times n_{dvv}$ matrix of doubles\ldots]].  Hessian
components (numeric values for second partial derivatives) are
listed by rows and separated by white space; in particular, they
can be spread across multiple
lines for readability.  Tags are not used and must not be present.
White space after the initial double bracket and before the final one
is optional, but none can appear within the double brackets.

The format of the numeric fields may be floating point or scientific
notation. In the latter case, acceptable exponent characters are
``\texttt{E}'' or ``\texttt{e.}'' A common problem when dealing with
Fortran programs is that a C++ read of a numeric field using
``\texttt{D}'' or ``\texttt{d}'' as the exponent (i.e., a double
precision value from Fortran) may fail or be truncated. In this case,
the ``\texttt{D}'' exponent characters must be replaced either through
modifications to the Fortran source or compiler flags or through a
separate post-processing step (e.g., using the UNIX \texttt{sed}
utility).

\section{Active Variables for Derivatives}\label{responses:active}

An important question for proper management of both gradient and
Hessian data is: if several different types of variables are used,
\emph{for which variables are response function derivatives needed?}
That is, how is $n_{dvv}$ determined?  The short answer is that the
derivative variables vector (DVV) specifies the set of variables to be
used for computing derivatives, and $n_{dvv}$ is the length of this
vector.  

In most cases, the DVV is defined directly from the set of active
continuous variables for the iterative method in use.  Since methods
operate over a subset, or view, of the variables that is active in the
iteration, it is this same set of variables for which derivatives are
most commonly computed.  Derivatives are never needed with respect to
any discrete variables (since these derivatives do not in general
exist) and the active continuous variables depend on view override
specifications, inference by response type, and inference by method
type, in that order, as described in Section~\ref{variables:mixed}.

%For optimization and calibration problems,
%the active continuous variables are the \emph{continuous design
%  variables} ($n_{dvv}=n_{cdv}$), since they are the variables the
%minimizer manipulates.  Similarly, for uncertainty quantification
%methods that use gradient and/or Hessian information, the active
%continuous variables are the \emph{continuous uncertain variables}
%($n_{dvv}=n_{cauv}$ for aleatory methods, $n_{dvv}=n_{ceuv}$ for
%epistemic methods, $n_{dvv}=n_{cauv}+n_{ceuv}$ for methods that handle
%both), with the exception of \texttt{all\_variables} mode.  And
%lastly, parameter study methods that are cataloging gradient and/or
%Hessian information do not draw a distinction among continuous
%variables; therefore, the active continuous variables are defined from
%\emph{all continuous variables} that are specified
%($n_{dvv}=n_{cdv}+n_{cauv}+n_{ceuv}+n_{csv}$).  Additional detail on
%these variables views is provided in Table~\ref{responses:active_tab}.

%\begin{table}
%\centering
%\caption{Variable views for different iterators.}
%\label{responses:active_tab}\vspace{2mm}
%\begin{tabular}{|c|c|c|}
%\hline
%\textbf{Method} & \textbf{Default Active view} & \textbf{Derivative variables} \\
%\hline
%branch and bound         & Merged Design   & $n_{cdv}+n_{ddiv}+n_{ddrv}$ \\
%\hline
%optimization,            & Mixed Design    & $n_{cdv}$ \\
%nonlinear least squares  &                 &           \\
%\hline
%sampling (standard mode) & Mixed Uncertain & $n_{cauv}+n_{ceuv}$ \\
%\hline
%local reliability,       & Mixed Aleatory Uncertain & $n_{cauv}$ \\
%global reliability (standard mode),  &              &            \\
%stochastic expansion (standard mode) &              &            \\
%\hline
%interval estimation,     & Mixed Epistemic Uncertain & $n_{ceuv}$ \\
%evidence                 &                           &            \\
%\hline
%parameter studies,       & Mixed All & $n_{cdv}+n_{cauv}+n_{ceuv}+n_{csv}$\\
%design of experiments,   &           & \\
%uncertainty quantification (all\_variables mode) & & \\
%\hline
%\end{tabular}
%\end{table}

In a few cases, derivatives are needed with respect to the
\emph{inactive} continuous variables.  This occurs for nested
iteration where a top-level iterator sets derivative requirements
(with respect to its active continuous variables) on the final
solution of the lower-level iterator (for which the top-level active
variables are inactive).  For example, in an uncertainty analysis
within a nested design under uncertainty algorithm, derivatives of the
lower-level response functions may be needed with respect to the
design variables, which are active continuous variables at the top
level but are inactive within the uncertainty quantification.  These
instances are the reason for the creation and inclusion of the DVV
vector --- to clearly indicate the variables whose partial derivatives
are needed.

In all cases, if the DVV is honored, then the correct derivative
components are returned.  In simple cases, such as optimization and
calibration studies that only specify design variables and for
nondeterministic analyses that only specify uncertain variables,
derivative component subsets are not an issue and the exact content of
the DVV may be safely ignored.
