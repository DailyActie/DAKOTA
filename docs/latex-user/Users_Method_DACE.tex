\chapter{Design of Experiments Capabilities}\label{dace}

\section{Overview}\label{dace:overview}

Classical design of experiments (DoE) methods and the more modern
design and analysis of computer experiments (DACE) methods are both
techniques which seek to extract as much trend data from a parameter
space as possible using a limited number of sample points. Classical
DoE techniques arose from technical disciplines that assumed some
randomness and nonrepeatability in field experiments (e.g.,
agricultural yield, experimental chemistry). DoE approaches such as
central composite design, Box-Behnken design, and full and fractional
factorial design generally put sample points at the extremes of the
parameter space, since these designs offer more reliable trend
extraction in the presence of nonrepeatability. DACE methods are
distinguished from DoE methods in that the nonrepeatability component
can be omitted since computer simulations are involved. In these
cases, space filling designs such as orthogonal array sampling and
Latin hypercube sampling are more commonly employed in order to
accurately extract trend information. Quasi-Monte Carlo sampling
techniques which are constructed to fill the unit hypercube with good
uniformity of coverage can also be used for DACE.

Dakota supports both DoE and DACE techniques. In common usage, only
parameter bounds are used in selecting the samples within the
parameter space. Thus, DoE and DACE can be viewed as special cases of
the more general probabilistic sampling for uncertainty quantification
(see following section), in which the DoE/DACE parameters are treated
as having uniform probability distributions. The DoE/DACE techniques
are commonly used for investigation of global response trends,
identification of significant parameters (e.g., main effects), and as
data generation methods for building response surface approximations.

Dakota includes several approaches sampling and design of experiments,
all implemented in included third-party software libraries. LHS
(Latin hypercube sampling)~\cite{Swi04} is a general-purpose sampling
package developed at Sandia that has been used by the DOE national
labs for several decades. DDACE (distributed design and analysis for
computer experiments) is a more recent package for computer
experiments developed at Sandia Labs~\cite{TonXX}. DDACE provides the
capability for generating orthogonal arrays, Box-Behnken designs,
Central Composite designs, and random designs. The FSUDace (Florida
State University's Design and Analysis of Computer Experiments)
package provides the following sampling techniques: quasi-Monte Carlo
sampling based on Halton or Hammersley sequences, and Centroidal
Voronoi Tessellation. Lawrence Livermore National Lab's PSUADE
(Problem Solving Environment for Uncertainty Analysis and Design
Exploration)~\cite{Ton05} includes several methods for model
exploration, but only the Morris screening method is exposed in
Dakota.

This chapter describes DDACE, FSUDace, and PSUADE, with a focus on
designing computer experiments. Latin Hypercube Sampling, also used in
uncertainty quantification, is discussed in Section~\ref{uq:sampling}.
\begin{comment}
The differences between sampling used in design of experiments and
sampling used in uncertainty quantification is discussed in more
detail in the following paragraphs. In brief, we consider design of
experiment methods to generate sets of uniform random variables on the
interval $[0,1]$. These sets are mapped to the lower/upper bounds of
the problem variables and then the response functions are evaluated at
the sample input points with the goal of characterizing the behavior
of the response functions over the input parameter ranges of
interest. Uncertainty quantification via LHS sampling, in contrast,
involves characterizing the uncertain input variables with probability
distributions such as normal, Weibull, triangular, etc., sampling from
the input distributions, and propagating the input uncertainties to
obtain a cumulative distribution function on the output. There is
significant overlap between design of experiments and sampling. Often,
both techniques can be used to obtain similar results about the
behavior of the response functions and about the relative importance
of the input variables.
\end{comment}

\section{Design of Computer Experiments}\label{dace:background}

What distinguishes design of {\em computer} experiments?  Computer
experiments are often different from physical experiments, such as
those performed in agriculture, manufacturing, or biology. In
physical experiments, one often applies the same \emph{treatment} or
\emph{factor level} in an experiment several times to get an
understanding of the variability of the output when that treatment is
applied. For example, in an agricultural experiment, several fields
(e.g., 8) may be subject to a low level of fertilizer and the same
number of fields may be subject to a high level of fertilizer to see
if the amount of fertilizer has a significant effect on crop
output. In addition, one is often interested in the variability of the
output within a treatment group: is the variability of the crop yields
in the low fertilizer group much higher than that in the high
fertilizer group, or not?

In physical experiments, the process we are trying to examine is stochastic:  
that is, the same treatment may result in different outcomes. 
By contrast, in computer experiments, often we have a deterministic code. 
If we run the code with a particular set of input parameters, the code 
will always produce the same output. There certainly are stochastic codes, 
but the main focus of computer experimentation has been on deterministic codes. 
Thus, in computer experiments we often do not have the need to do replicates 
(running the code with the exact same input parameters several times to see 
differences in outputs). Instead, a major concern in computer experiments is 
to create an experimental design which can sample a high-dimensional space 
in a representative way with a minimum number of samples. 
The number of factors or parameters that we wish to explore in computer 
experiments is usually much higher than physical experiments. 
In physical experiments, one may be interested in varying a few parameters, 
usually five or less, while in computer experiments we often have 
dozens of parameters of interest. Choosing the levels of these parameters 
so that the samples adequately explore the input space is a challenging 
problem. There are many experimental designs and sampling methods 
which address the issue of adequate and representative sample selection. 
%Classical experimental designs which are often used in physical experiments 
%include Central Composite designs and Box-Behnken designs.

There are many goals of running a computer experiment: one may want to 
explore the input domain or the design space and get a better understanding 
of the range in the outputs for a particular domain. Another objective is 
to determine which inputs have the most influence on the output, or how 
changes in the inputs change the output. This is usually called 
\emph{sensitivity analysis}. 
%Another goal is to compare the relative 
%importance of model input uncertainties on the uncertainty in the model 
%outputs, \emph{uncertainty analysis}. 
Another goal is to use the 
sampled input points and their corresponding output to create a 
\emph{response surface approximation} for the computer code. The response 
surface approximation (e.g., a polynomial regression model, a 
Gaussian-process/Kriging model, a neural net) can then be used to emulate 
the computer code. 
Constructing a response surface approximation is particularly important 
for applications where running a computational model is extremely expensive:  
the computer model may take 10 or 20 hours to run on a high performance 
machine, whereas the response surface model may only take a few seconds. 
Thus, one often optimizes the response surface model or uses it within a 
framework such as surrogate-based optimization. Response surface models 
are also valuable in cases where the gradient (first derivative) and/or 
Hessian (second derivative) information required by optimization techniques 
are either not available, expensive to compute, or inaccurate because the 
derivatives are poorly approximated or the function evaluation is itself 
noisy due to roundoff errors. Furthermore, many optimization methods 
require a good initial point to ensure fast convergence or to converge to 
good solutions (e.g. for problems with multiple local minima). Under these 
circumstances, a good design of computer experiment framework coupled with 
response surface approximations can offer great advantages. 

In addition to the sensitivity analysis and response 
surface modeling mentioned above, we also may want to do 
\emph{uncertainty quantification} on a computer model. 
Uncertainty quantification (UQ) refers to taking a particular set of 
distributions on the inputs, and propagating them through the model to 
obtain a distribution on the outputs. For example, if input parameter A 
follows a normal with mean 5 and variance 1, the computer produces a random 
draw from that distribution. If input parameter B follows a weibull 
distribution with alpha = 0.5 and beta = 1, the computer produces a random 
draw from that distribution. When all of the uncertain variables have 
samples drawn from their input distributions, we run the model with the 
sampled values as inputs. We do this repeatedly to build up a distribution 
of outputs. We can then use the cumulative distribution function of the 
output to ask questions such as:  what is the probability that the output is 
greater than 10?   What is the 99th percentile of the output?  

Note that sampling-based uncertainty quantification and design of
computer experiments are very similar. \emph{There is significant
overlap} in the purpose and methods used for UQ and for DACE. We have
attempted to delineate the differences within Dakota as follows: we
use the methods DDACE, FSUDACE, and PSUADE primarily for design of
experiments, where we are interested in understanding the main effects
of parameters and where we want to sample over an input domain to
obtain values for constructing a response surface. We use the
nondeterministic sampling methods \texttt{(sampling)} for
uncertainty quantification, where we are propagating specific input
distributions and interested in obtaining (for example) a cumulative
distribution function on the output. If one has a problem with no
distributional information, we recommend starting with a design of
experiments approach. Note that DDACE, FSUDACE, and PSUADE currently
do \emph{not} support distributional information: they take an upper
and lower bound for each uncertain input variable and sample within
that. The uncertainty quantification methods in
\texttt{sampling} (primarily Latin Hypercube sampling) offer the
capability to sample from many distributional types. The distinction
between UQ and DACE is somewhat arbitrary: both approaches often can
yield insight about important parameters and both can determine sample
points for response surface approximations.

Three software packages are available in Dakota for design of computer
experiments, DDACE (developed at Sandia Labs), FSUDACE (developed at
Florida State University), and PSUADE (LLNL).

\section{DDACE}\label{dace:ddace}

The Distributed Design and Analysis of Computer Experiments (DDACE)
package includes both classical design of experiments
methods~\cite{TonXX} and stochastic sampling methods. The classical
design of experiments methods in DDACE are central composite design
(CCD) and Box-Behnken (BB) sampling. A grid-based sampling
(full-factorial) method is also available. The stochastic methods are
orthogonal array sampling~\cite{Koe96} (which permits main effects
calculations), Monte Carlo (random) sampling, Latin hypercube
sampling, and orthogonal array-Latin hypercube sampling. While DDACE
LHS supports variables with normal or uniform distributions, only
uniform are supported through Dakota. Also DDACE does not allow
enforcement of user-specified correlation structure among the
variables.

The sampling methods in DDACE can be used alone or in conjunction with
other methods. For example, DDACE sampling can be used with both the
surrogate-based optimization strategy and the optimization under
uncertainty strategy. See Figure~\ref{adv_models:figure09} for an
example of how the DDACE settings are used in Dakota.

%More information on DDACE is available on the web at:
%\url{http://csmr.ca.sandia.gov/projects/ddace}

The following sections provide more detail about the sampling 
methods available for design of experiments in DDACE. 

\subsection{Central Composite Design}\label{dace:ccd}

A Box-Wilson Central Composite Design, commonly called a central
composite design (CCD), contains an embedded factorial or fractional
factorial design with center points that is augmented with a group of
'star points' that allow estimation of curvature. If the distance
from the center of the design space to a factorial point is $\pm$1
unit for each factor, the distance from the center of the design space
to a star point is $\pm\alpha$ with $\mid\alpha\mid > 1$. The precise
value of $\alpha$ depends on certain properties desired for the design and on
the number of factors involved. The CCD design is specified in Dakota
with the method command \texttt{dace central\_composite}.

As an example, with two input variables or factors, each having two 
levels, the factorial design is shown in Table~\ref{dace:table01} . 

\begin{table}[ht]
 \caption{Simple Factorial Design}
 \label{dace:table01}	
 \begin{center}
  \begin{tabular}{c|c}
  \hline
  Input 1            & Input 2         \\ \hline \hline 
  -1                 & -1             \\ \hline 
  -1                 & +1           \\ \hline
  +1                 & -1      \\ \hline
  +1                 & +1        \\ \hline
  \end{tabular}
\end{center}
\end{table}


With a CCD, the design in Table~\ref{dace:table01} would be augmented 
with the following points shown in Table~\ref{dace:table02} 
if $\alpha$ = 1.3. These points define a circle around the original  
factorial design.

\begin{table}[ht]
 \caption{Additional Points to make the factorial design a CCD}
 \label{dace:table02}
 \begin{center}
  \begin{tabular}{c|c}
  \hline
  Input 1            & Input 2         \\ \hline \hline 
  0                 & +1.3             \\ \hline 
  0                 & -1.3           \\ \hline
  1.3                 & 0     \\ \hline
  -1.3                 & 0       \\ \hline
  0                  & 0          \\ \hline
  \end{tabular}
\end{center}
\end{table}

Note that the number of sample points specified in a CCD,\texttt{samples},
is a function of the number of variables in the problem: 

\[
samples = 1 + 2*NumVar + 2^{NumVar}
\]

\subsection{Box-Behnken Design}\label{dace:bb}

The Box-Behnken design is similar to a Central Composite design, with
some differences. The Box-Behnken design is a quadratic design in
that it does not contain an embedded factorial or fractional factorial
design. In this design the treatment combinations are at the midpoints
of edges of the process space and at the center, as compared with CCD
designs where the extra points are placed at 'star points' on a circle
outside of the process space. Box-Behken designs are rotatable (or
near rotatable) and require 3 levels of each factor. The designs have
limited capability for orthogonal blocking compared to the central
composite designs. Box-Behnken requires fewer runs than CCD for 3
factors, but this advantage goes away as the number of factors
increases. The Box-Behnken design is specified in Dakota with the
method command \texttt{dace box\_behnken}.

Note that the number of sample points specified in a Box-Behnken design,
\texttt{samples}, is a function of the number of variables in the problem: 

\[
samples = 1 + 4*NumVar + (NumVar-1)/2
\]

\subsection{Orthogonal Array Designs}\label{dace:oas}

Orthogonal array (OA) sampling is a widely used technique for 
running experiments and systematically testing factor effects~\cite{Hed99}. 
An orthogonal array sample can be described as a 4-tuple $(m,n,s,r)$,
where $m$ is the number of sample points, $n$ is the number of input variables, 
$s$ is the number of symbols, and $r$ is the strength of the orthogonal array. 
The number of sample points, $m$, must be a multiple of the number of symbols, 
$s$. The number of symbols refers to the number of levels per input variable. 
The strength refers to the number of columns where we are guaranteed to 
see all the possibilities an equal number of times.

For example, Table~\ref{dace:table03} shows an orthogonal array of strength 2 for $m$ = 8, with 7 variables:

\begin{table}[ht]
 \caption{Orthogonal Array for Seven Variables}
 \label{dace:table03}
 \begin{center}
  \begin{tabular}{c|c|c|c|c|c|c}
  \hline
  Input 1 & Input 2 & Input 3 & Input 4 & Input 5 & Input 6 & Input 7\\ \hline \hline 
0 & 	0 &	0 &	0 & 	0 &	0 &	0  \\ \hline
0 &	0 &	0 &	1 &	1 &	1 &	1   \\ \hline
0 &	1 & 	1 & 	0 & 	0 &	1 &	1   \\ \hline
0 &	1 &	1 &	1 &	1 &	0 &	0    \\ \hline
1 &	0 &	1 &	0 &	1 &	0 &	1   \\ \hline
1 &	0 &	1 &	1 &	0 &	1 &	0 \\ \hline
1 &	1 &	0 &	0 &	1 &	1 &	0 \\ \hline
1 &	1 &	0 &	1 &	0 &	0 &	1 \\ \hline

  \end{tabular}
\end{center}
\end{table}


If one picks any two columns, say the first and the third, note that 
each of the four possible rows we might see there,
0 0,       0 1,       1 0,       1 1,
appears exactly the same number of times, twice in this case.

DDACE creates orthogonal arrays of strength 2. Further, 
the OAs generated by DDACE do not treat the factor levels as one 
fixed value (0 or 1 in the above example). Instead, once a level 
for a variable is determined in the array,  DDACE 
samples a random variable from within that level.
The orthogonal array design is specified in 
Dakota with the method command \texttt{dace oas}. 

The orthogonal array method in DDACE is the only method that 
allows for the calculation of main effects, specified with the 
command \texttt{main\_effects}. Main effects is a sensitivity analysis 
method which identifies the input variables that have the most 
influence on the output. In main effects, the idea is to look 
at the mean of the response function when variable A (for example) 
is at level 1 vs. when variable A is at level 2 or level 3. 
If these mean responses of the output are statistically significantly 
different at different levels of variable A, this is an indication that 
variable A has a significant effect on the response. 
The orthogonality of the columns is critical in performing 
main effects analysis, since the column orthogonality means 
that the effects of the other variables 'cancel out' when 
looking at the overall effect from one variable at its different 
levels. There are ways of developing orthogonal arrays to calculate 
higher order interactions, such as two-way interactions (what 
is the influence of Variable A * Variable B on the output?), but this is 
not available in DDACE currently. At present, one way interactions 
are supported in the calculation of orthogonal array main effects within DDACE.
The main effects are presented as a series of ANOVA tables. 
For each objective function and constraint, the decomposition of variance 
of that objective or constraint is presented as a function of the 
input variables. The p-value in the ANOVA table is used to indicate 
if the input factor is significant. The p-value is the probability that 
you would have obtained samples more extreme than you did if the input 
factor has no effect on the response. For example, if you set a level 
of significance at 0.05 for your p-value, and the actual p-value is 0.03, 
then the input factor has a significant effect on the response. 

\subsection{Grid Design}\label{dace:grid}

In a grid design, a grid is placed over the input variable space. 
This is very similar to a multi-dimensional parameter study where 
the samples are taken over a set of partitions on each variable 
(see Section~\ref{ps:multidimensional}). The main difference is 
that in grid sampling, a small random perturbation is added 
to each sample value so that the grid points are not on a perfect grid. 
This is done to help capture certain features in the output such as periodic
functions. A purely structured grid, with the samples exactly on the grid 
points, has the disadvantage of not being able to capture important features 
such as periodic functions with relatively high frequency (due to aliasing). 
Adding a random perturbation to the grid samples helps remedy this problem.

Another disadvantage with grid sampling is that the number of sample points 
required depends exponentially on the input dimensions. In grid sampling, 
the number of samples is the number of symbols (grid partitions) raised 
to the number of variables. For example, if there are 2 variables, each 
with 5 partitions, the number of samples would be $5^2$. In this 
case, doubling the number of variables squares the sample size. 
The grid design is specified in 
Dakota with the method command \texttt{dace grid}.

\subsection{Monte Carlo Design}\label{dace:mc}

Monte Carlo designs simply involve pure Monte-Carlo random sampling 
from uniform distributions between the lower and upper bounds on each 
of the input variables. Monte Carlo designs, specified by 
\texttt{dace random}, are a way to generate a set of random samples 
over an input domain.

\subsection{LHS Design}\label{dace:lhs}

DDACE offers the capability to generate Latin Hypercube designs. 
For more information on Latin Hypercube sampling, see 
Section~\ref{uq:sampling}. Note that the version of LHS in DDACE 
generates uniform samples (uniform between the variable bounds). 
The version of LHS offered with nondeterministic sampling can generate 
LHS samples according to a number of distribution types, including 
normal, lognormal, weibull, beta, etc. To specify the DDACE version 
of LHS, use the method command \texttt{dace lhs}.

\subsection{OA-LHS Design}\label{dace:oalhs}

DDACE offers a hybrid design which is combination of an orthogonal
array and a Latin Hypercube sample. This design is specified with the
method command \texttt{dace oa\_lhs}. This design has the advantages
of both orthogonality of the inputs as well as stratification of the
samples (see~\cite{Owe92}).

\section{FSUDace}\label{dace:fsudace}

The Florida State University Design and Analysis of Computer
Experiments (FSUDace) package provides quasi-Monte Carlo sampling
(Halton and Hammersley) and Centroidal Voronoi Tessellation (CVT)
methods. All three methods natively generate sets of uniform random
variables on the interval $[0,1]$ (or in Dakota, on user-specified
uniform intervals).

The quasi-Monte Carlo and CVT methods are designed with the goal of
low discrepancy. Discrepancy refers to the nonuniformity of the sample
points within the unit hypercube. Low discrepancy sequences tend to
cover the unit hypercube reasonably uniformly. Quasi-Monte Carlo
methods produce low discrepancy sequences, especially if one is
interested in the uniformity of projections of the point sets onto
lower dimensional faces of the hypercube (usually 1-D: how well do the
marginal distributions approximate a uniform?) CVT does very well
volumetrically: it spaces the points fairly equally throughout the
space, so that the points cover the region and are isotropically
distributed with no directional bias in the point placement. There are
various measures of volumetric uniformity which take into account the
distances between pairs of points, regularity measures, etc. Note that
CVT does not produce low-discrepancy sequences in lower dimensions,
however: the lower-dimension (such as 1-D) projections of CVT can have
high discrepancy.

The quasi-Monte Carlo sequences of Halton and Hammersley are
deterministic sequences determined by a set of prime bases.
A Halton design is specified in Dakota with the method command 
\texttt{fsu\_quasi\_mc halton}, and the Hammersley design is 
specified with the command \texttt{fsu\_quasi\_mc hammersley}.
For more details about the input specification, see the Reference Manual.
CVT points tend to arrange themselves in a pattern
of cells that are roughly the same shape. To produce CVT
points, an almost arbitrary set of initial points is chosen, and then
an internal set of iterations is carried out. These iterations
repeatedly replace the current set of sample points by an estimate of
the centroids of the corresponding Voronoi subregions~\cite{Du99}.
A CVT design is specified in Dakota with the method command
\texttt{fsu\_cvt}.

The methods in FSUDace are useful for design of experiments because 
they provide good coverage of the input space, thus allowing global 
sensitivity analysis. 

\section{PSUADE MOAT}\label{dace:psuade}

PSUADE (Problem Solving Environment for Uncertainty Analysis and
Design Exploration) is a Lawrence Livermore National Laboratory tool
for metamodeling, sensitivity analysis, uncertainty quantification,
and optimization. Its features include non-intrusive and parallel
function evaluations, sampling and analysis methods, an integrated
design and analysis framework, global optimization, numerical
integration, response surfaces (MARS and higher order regressions),
graphical output with Pgplot or Matlab, and fault
tolerance~\cite{Ton05}. Dakota includes a prototype interface to its
Morris One-At-A-Time (MOAT) screening method, a valuable tool for
global sensitivity (including interaction) analysis.

The Morris One-At-A-Time method, originally proposed by
M.~D. Morris~\cite{Mor91}, is a screening method, designed to explore
a computational model to distinguish between input variables that have
negligible, linear and additive, or nonlinear or interaction effects
on the output. The computer experiments performed consist of
individually randomized designs which vary one input factor at a time
to create a sample of its elementary effects.

With MOAT, each dimension of a $k-$dimensional input space is
uniformly partitioned into $p$ levels, creating a grid of $p^k$ points
${\bf x} \in \mathbb{R}^k$ at which evaluations of the model $y({\bf
x})$ might take place. An elementary effect corresponding to input
$i$ is computed by a forward difference
\begin{equation}
d_i({\bf x}) = \frac{y({\bf x} + \Delta {\bf e}_i) - y({\bf x})}{\Delta},
\end{equation}
where $e_i$ is the $i^{\mbox{\scriptsize th}}$ coordinate vector, and
the step $\Delta$ is typically taken to be large (this is not intended
to be a local derivative approximation). In the present
implementation of MOAT, for an input variable scaled to $[0,1]$,
$\Delta = \frac{p}{2(p-1)}$, so the step used to find elementary
effects is slightly larger than half the input range.

The distribution of elementary effects $d_i$ over the input space
characterizes the effect of input $i$ on the output of interest.
After generating $r$ samples from this distribution, their mean,
\begin{equation}
\mu_i = \frac{1}{r}\sum_{j=1}^{r}{d_i^{(j)}},
\end{equation}
modified mean
\begin{equation}
\mu_i^* = \frac{1}{r}\sum_{j=1}^{r}{|d_i^{(j)}|},
\end{equation}
(using absolute value) and standard deviation
\begin{equation}
\sigma_i = \sqrt{ \frac{1}{r}\sum_{j=1}^{r}{ \left(d_i^{(j)} - \mu_i
\right)^2} }
\end{equation}
are computed for each input $i$. The mean and modified mean give an
indication of the overall effect of an input on the output. Standard
deviation indicates nonlinear effects or interactions, since it is an
indicator of elementary effects varying throughout the input space.

The MOAT method is selected with method keyword {\tt psuade\_moat} as
shown in the sample Dakota input deck in Figure~\ref{FIG:moat_input}.
The number of samples ({\tt samples}) must be a positive integer
multiple of (number of continuous design variables $k$ + 1) and will
be automatically adjusted if misspecified. The number of partitions
({\tt partitions}) applies to each variable being studied and must be
odd (the number of MOAT levels $p$ per variable is partitions + 1,
similar to Dakota multidimensional parameter studies). This will also
be adjusted at runtime as necessary. Finite user-specified lower and
upper bounds are required and will be scaled as needed by the method.
For more information on use of MOAT sampling, see the Morris example
in Section~\ref{additional:morris}, or Saltelli, et al.~\cite{Sal04}.

\begin{figure}
  \centering \begin{bigbox} \begin{small}
  \verbatimtabinput[8]{morris_ps_moat.in} \end{small} \end{bigbox}
\caption{Dakota input file showing the Morris One-at-a-Time method --
see {\tt Dakota/examples/users/morris\_ps\_moat.in} }
\label{FIG:moat_input}
\end{figure}

\section{Sensitivity Analysis}\label{dace:sa}

\subsection{Sensitivity Analysis Overview}\label{dace:sa:overview}

In many engineering design applications, sensitivity analysis
techniques and parameter study methods are useful in identifying which
of the design parameters have the most influence on the response
quantities. This information is helpful prior to an optimization study
as it can be used to remove design parameters that do not strongly
influence the responses. In addition, these techniques can provide
assessments as to the behavior of the response functions (smooth or
nonsmooth, unimodal or multimodal) which can be invaluable in
algorithm selection for optimization, uncertainty quantification, and
related methods. In a post-optimization role, sensitivity information
is useful is determining whether or not the response functions are
robust with respect to small changes in the optimum design point.

In some instances, the term sensitivity analysis is used in a local
sense to denote the computation of response derivatives at a point.
These derivatives are then used in a simple analysis to make design
decisions. Dakota supports this type of study through numerical
finite-differencing or retrieval of analytic gradients computed within
the analysis code. The desired gradient data is specified in the
responses section of the Dakota input file and the collection of this
data at a single point is accomplished through a parameter study
method with no steps. This approach to sensitivity analysis should be
distinguished from the activity of augmenting analysis codes to
internally compute derivatives using techniques such as direct or
adjoint differentiation, automatic differentiation (e.g., ADIFOR), or
complex step modifications. These sensitivity augmentation activities
are completely separate from Dakota and are outside the scope of this
manual. However, once completed, Dakota can utilize these analytic
gradients to perform optimization, uncertainty quantification, and
related studies more reliably and efficiently.

In other instances, the term sensitivity analysis is used in a more
global sense to denote the investigation of variability in the
response functions. Dakota supports this type of study through
computation of response data sets (typically function values only, but
all data sets are supported) at a series of points in the parameter
space. The series of points is defined using either a vector, list,
centered, or multidimensional parameter study method. For example, a
set of closely-spaced points in a vector parameter study could be used
to assess the smoothness of the response functions in order to select
a finite difference step size, and a set of more widely-spaced points
in a centered or multidimensional parameter study could be used to
determine whether the response function variation is likely to be
unimodal or multimodal. See Chapter~\ref{ps} for additional
information on these methods. These more global approaches to
sensitivity analysis can be used to obtain trend data even in
situations when gradients are unavailable or unreliable, and they are
conceptually similar to the design of experiments methods and sampling
approaches to uncertainty quantification described in the following
sections.

\subsection{Assessing Sensitivity with DACE}\label{dace:sa:assessing}

Like parameter studies (see Chapter~\ref{ps}), the DACE techniques are
useful for characterizing the behavior of the response functions of
interest through the parameter ranges of interest. In addition to
direct interrogation and visualization of the sampling results, a
number of techniques have been developed for assessing the parameters
which are most influential in the observed variability in the response
functions. One example of this is the well-known technique of scatter
plots, in which the set of samples is projected down and plotted
against one parameter dimension, for each parameter in turn. Scatter
plots with a uniformly distributed cloud of points indicate parameters
with little influence on the results, whereas scatter plots with a
defined shape to the cloud indicate parameters which are more
significant. Related techniques include analysis of variance
(ANOVA)~\cite{Mye95} and main effects analysis, in which the parameters
which have the greatest influence on the results are identified from
sampling results. Scatter plots and ANOVA may be accessed through
import of Dakota tabular results (see Section~\ref{output:tabular})
into external statistical analysis programs such as S-plus, Minitab,
etc.

Running any of the design of experiments or sampling methods allows
the user to save the results in a tabular data file, which then can be
read into a spreadsheet or statistical package for further analysis.
In addition, we have provided some functions to help determine the
most important variables.

We take the definition of uncertainty analysis from~\cite{Sal04}: 
``The study of how uncertainty in the output of a model can be 
apportioned to different sources of uncertainty in the model input.''

As a default, Dakota provides correlation analyses when running LHS.
Correlation tables are printed with the simple, partial, and rank
correlations between inputs and outputs. These can be useful to get a
quick sense of how correlated the inputs are to each other, and how
correlated various outputs are to inputs. The correlation analyses are
explained further in Chapter~\ref{uq:sampling}.

We also have the capability to calculate sensitivity indices through
Variance-based Decomposition (VBD). Variance-based decomposition 
is a global sensitivity method that summarizes how the uncertainty 
in model output can be apportioned to uncertainty in individual 
input variables. VBD uses two primary measures, the main effect 
sensitivity index $S_{i}$ and the total effect index $T_{i}$. The 
main effect sensitivity 
index corresponds to the fraction of the uncertainty in the output, $Y$, 
that can be attributed to input $x_{i}$ alone. The total effects index 
corresponds to the fraction of the uncertainty in 
the output, $Y$, that can be attributed to input $x_{i}$ and its 
interactions with other variables. The main effect sensitivity index
compares the variance of the conditional expectation
$Var_{x_{i}}[E(Y|x_{i})]$ against the total variance $Var(Y)$.
Formulas for the indices are: 

\begin{equation}
S_{i}=\frac{Var_{x_{i}}[E(Y|x_{i})]}{Var(Y)} \label{eq:VBD_Si}
\end{equation}

and 
\begin{equation}
T_{i}=\frac{E(Var(Y|x_{-i}))}{Var(Y)}=\frac{Var(Y)-Var(E[Y|x_{-i}])}{Var(Y)} \label{eq:VBD_Ti}
\end{equation}

where $Y=f({\bf x})$ and ${x_{-i}=(x_{1},...,x_{i-1},x_{i+1},...,x_{m})}$.

The calculation of $S_{i}$ and $T_{i}$ requires the evaluation of 
m-dimensional integrals which are typically approximated by Monte-Carlo 
sampling. More details on the
calculations and interpretation of the sensitivity indices can be
found in~\cite{Sal04}. In Dakota version 5.1, we have 
improved calculations for the calculation of the $S_{i}$ and $T_{i}$ 
indices when using sampling. The implementation details of these 
calculatiosn are provided in~\cite{Weirs10}. 
VBD can be specified for any of the sampling or DACE methods using the 
command \texttt{variance\_based\_decomposition}.
Note that VBD is extremely computationally intensive when using sampling 
since replicated sets of sample values are evaluated. If the
user specified a number of samples, $N$, and a number of
nondeterministic variables, $M$, variance-based decomposition
requires the evaluation of $N(M+2)$ samples. To obtain
sensitivity indices that are reasonably accurate, we recommend that
$N$, the number of samples, be at least one hundred and
preferably several hundred or thousands. Because of the computational
cost, variance-based decomposition is turned off as a default
for sampling or DACE. Another alternative, however, is to obtain 
these indices using one of the stochastic expansion methods 
described in Section~\ref{uq:expansion}. The calculation 
of the indices using expansion methods is much more efficient 
since the VBD indices are analytic functions of the coefficients 
in the stochastic expansion. The paper by Weirs et al.~\cite{Weirs10}
compares different methods for calculating the sensitivity 
indices for nonlinear problems with significant interaction effects.

In terms of interpretation of the sensitivity indices, a larger value of
the sensitivity index, $S_{i}$,
means that the uncertainty in the input variable $i$ has a
larger effect on the variance of the output. Note that the 
sum of the main effect indices will be less than or equal to one. 
If the sum of the main effect indices is much less than one, 
it indicates that there are significant two-way, three-way, or higher
order interactions that contribute significantly to the variance. 
There is no requirement that the sum of the total effect indices 
is one:  in most cases, the sum of the total effect indices will be 
greater than one. An example of the Main and Total effects 
indices as calculated by Dakota using sampling is shown 
in Figure~\ref{fig:dace:vbd}

\begin{figure}[ht!]
\centering
\begin{bigbox}
\begin{small}
\begin{verbatim}
Global sensitivity indices for each response function:
response_fn_1 Sobol indices:
                                  Main             Total
                      4.7508913283e-01  5.3242162037e-01 uuv_1
                      3.8112392892e-01  4.9912486515e-01 uuv_2
\end{verbatim}
\end{small}
\end{bigbox}
\caption{Dakota output for Variance-based Decomposition} 
\label{fig:dace:vbd}
\end{figure}

Finally, we have the capability to calculate a set of quality metrics 
for a particular input sample. These quality metrics measure 
various aspects relating to the volumetric spacing of the samples: 
are the points equally spaced, do they cover the region, are they 
isotropically distributed, do they have directional bias, etc.? 
The quality metrics are explained in more detail in the Reference Manual.

\section{DOE Usage Guidelines}\label{dace:usage}

Parameter studies, classical design of experiments (DOE),
design/analysis of computer experiments (DACE), and sampling methods
share the purpose of exploring the parameter space. When a global
space-filling set of samples is desired, then the DOE, DACE, and
sampling methods are recommended. These techniques are useful for
scatter plot and variance analysis as well as surrogate model
construction. 

The distinction between DOE and DACE methods is that the
former are intended for physical experiments containing an element of
nonrepeatability (and therefore tend to place samples at the extreme
parameter vertices), whereas the latter are intended for repeatable
computer experiments and are more space-filling in nature. 

The distinction between DOE/DACE and sampling is drawn based on the
distributions of the parameters. DOE/DACE methods typically assume
uniform distributions, whereas the sampling approaches in Dakota
support a broad range of probability distributions. 

To use \texttt{sampling} in design of experiments mode (as opposed to 
uncertainty quantification mode), the \texttt{all\_variables} flag
should be included in the method specification of the Dakota input
file.

Design of experiments method selection recommendations are summarized
in Table~\ref{dace:usage:table}.

\begin{table}[hbp]
\centering
\caption{Guidelines for selection of parameter study, DOE, DACE, and
sampling methods.}
\label{dace:usage:table}\vspace{2mm}
\begin{tabular}{|c|c|c|}
\hline
\textbf{Method} & \textbf{Applications} & \textbf{Applicable Methods} \\
\textbf{Classification} & & \\
\hline
parameter study & sensitivity analysis,                       & centered\_parameter\_study, \\
                & directed parameter space investigations     & list\_parameter\_study, \\
                &                                             & multidim\_parameter\_study, \\
                &                                             & vector\_parameter\_study \\
                   &                                             & \\
\hline
classical design & physical experiments                       & dace (box\_behnken, \\
of experiments   & (parameters are uniformly distributed)     & central\_composite) \\
                   &                                             & \\
\hline
design of computer & variance analysis,                       & dace (grid, random,
                                                                oas, lhs, oa\_lhs), \\
experiments        & space filling designs                    & fsu\_quasi\_mc (halton, 
                                                                hammersley), \\
                   & (parameters are uniformly distributed)   & fsu\_cvt, psuade\_moat \\
                   &                                             & \\
\hline
sampling           & space filling designs                    & sampling (Monte Carlo or LHS) \\
                   & (parameters have general probability distributions) & with all\_variables flag \\
                   &                                             & \\
\hline
\end{tabular}
\end{table}
